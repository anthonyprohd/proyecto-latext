\documentclass[11pt]{article}
\usepackage[spanish]{babel}
\usepackage{graphicx}
%opening
\title{Medida e Incerteza en la Medida}
\author{Autor: Anthony Joel Patzán Sián.  Carné: 202300469}
\date{\today}
\begin{document}

\maketitle
\section{¿Que es una medida?}
Una medición es comparar la cantidad desconocida que queremos determinar y una cantidad conocida de la misma magnitud, que elegimos como unidad. Al resultado de medir se le denomina medida.

\begin{figure}[h]
	\includegraphics[width=0.4\linewidth]{figura1}
	\caption{}
	\label{fig:figura1}
\end{figure}

\section{¿Que es la incerteza en la medida?}
Nos da una indicación de la calidad del resultado, de manera tal que el usuario puede apreciar su confiabilidad y de este modo se pueden comparar sus resultados contra alguna especificación o norma.



\end{document}
